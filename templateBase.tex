\documentclass{article}
\usepackage{amsmath, fullpage}

\begin{document}

\title{The line of best fit via transformations}
\author{David G Radcliffe}
%\date{\vspace{-5ex}}
\maketitle
\thispagestyle{empty}

In this note, we will show how transformations can be used to obtain
a radically simple derivation of the equation of the line of best fit.
Our approach also gives a simple geometric interpretation of the Pearson
correlation coefficient.

Given a sequence of $n$ points in the plane $(X_1, Y_1), \ldots, (X_n, Y_n)$
we seek the linear equation $y = a + bx$ that approximates the points
as closely as possible, in the sense that the sum of the squared residuals
$E = \sum_{i=1}^n (Y_i - a - bX_i)^2$ is minimized.

We assume that not all of the points lie on a single horizontal or vertical
line. In that case, we can apply a \emph{transformation} to the points
so that $\sum x_i = \sum y_i = 0$ and $\sum x_i^2 = \sum y_i^2 = 1$.
The transformation is defined by
$$
x_i = \frac{X_i - \overline{X}}{\sqrt{\sum (X_i - \overline{X})^2}}\quad\text{and}\quad
y_i = \frac{Y_i - \overline{Y}}{\sqrt{\sum (Y_i - \overline{Y})^2}}.
$$

This transformation is linear, so it maps lines to lines.
If we transform a line fitted to the data,
the sum of squared residuals is multiplied by a positive constant factor.
Therefore, the transformation preserves the line of best fit.


Let $r = \sum x_i y_i$. Then
\begin{align*}
E &= \sum (y_i - a - bx_i)^2 \\
&= \sum (y_i^2 + a^2 + b^2 x_i^2 - 2ay_i - 2bx_iy_i + 2abx_i) \\
&= \sum y_i^2 + \sum a^2 + \sum b^2 x_i^2 
 - \sum 2ay_i - \sum 2bx_iy_i + \sum 2abx_i \\
&= 1 + na^2 + b^2 - 2br \\
&= (1-r^2) + na^2 + (b - r)^2\ .
\end{align*}

The sum is minimized when $a = 0$ and $b = r$, so the line of best fit is
$y = rx$. What a simple equation!
Unfortunately, the equation is a bit messier when expressed in terms of the
original variables.

\begin{align*}
\frac{y - \overline{Y}}{\sqrt{\sum (Y_i - \overline{Y})^2}}
&= \left(
     \frac{\sum (X_i - \overline{X}) (Y_i - \overline{Y})}
          {\sqrt{\sum (X_i - \overline{X})^2 \sum (Y_i - \overline{Y})^2}}
   \right)
   \left(
     \frac{x - \overline{X}}
          {\sqrt{\sum (X_i - \overline{X})^2}}
   \right)\\
y - \overline{Y} &=
\left(
     \frac{\sum (X_i - \overline{X}) (Y_i - \overline{Y})}
          {\sum (X_i - \overline{X})^2}
   \right)
   (x - \overline{X})\ .
\end{align*}

Note that $r$ is the Pearson correlation coefficient of the sample.
This shows that the correlation coefficient can be interpreted geometrically
as the slope of the line of best fit when the $x$ and $y$ values are standardized.
\end{document}
